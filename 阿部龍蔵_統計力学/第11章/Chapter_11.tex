\documentclass[dvipdfmx,11pt]{beamer}
\usetheme{Warsaw}
\usecolortheme[RGB={32, 32, 96}]{structure}
\usefonttheme{professionalfonts}
\setbeamertemplate{navigation symbols}{}
\AtBeginSection[]{
    \frame{\tableofcontents[currentsection, hideallsubsections]}        %目次スライド
}

\usepackage{graphicx}
\usepackage{amsmath, amssymb}
\usepackage{bm}
\usepackage{multicol}
\usepackage{framed}
\usepackage{mathrsfs}

\usepackage{grffile}
\usepackage{here}
\usepackage{tikz}

\title{第11章\\グリーン関数に対する摂動論}
\author{Ryoi Ohashi}
\date{July 16, 2018}
\institute{Department of Applied Physics, Nagoya University}


\begin{document}

% \frame{\titlepage}
\begin{frame}[plain]
    \maketitle
\end{frame}

\begin{frame}\frametitle{目次}
    \setcounter{tocdepth}{1}
    \tableofcontents
\end{frame}

\begin{frame}\frametitle{復習}
    ハミルトニアンが$H=H_0+H^{'}$, $\mathscr{H}=H+\mu N$のとき
    \begin{block}{1体のグリーン関数}
        \begin{align}
            G_r[u,u'] = -<TA_r(u)A_r^{\dagger}(u')>
        \end{align}
    \end{block}
    \setcounter{equation}{2}
    \begin{itemize}
        \item $r$は状態を表す指数\\
        \item $A_r^{(\dagger)}(u)$は$a_r^{(\dagger)}$のハイゼンベルグ表示\\
        \item $<\cdots>$は大正準集団に対する平均\\
        \begin{align}
            <\cdots>={\rm tr}(\cdots e^{-\beta\mathscr{H}})/{\rm tr}(e^{-\beta\mathscr{H}})
        \end{align}
        
    \end{itemize}
\end{frame}

\section{$T$指数関数および$T$記号の性質}
\begin{frame}
    hello
\end{frame}
\subsection{$T$記号の性質}

\section{グリーン関数に対する表式}
\begin{frame}
    hello
\end{frame}

\section{グリーン関数に対するファインマン図形}
\begin{frame}
    hello
\end{frame}

\subsection{一般項に対する考察}
\begin{frame}
    hello
\end{frame}
\subsection{$G_r(i\omega_l)$の計算の規則}

\section{自己エネルギー(self-energy)}
\begin{frame}
    hello
\end{frame}

\section{電子ガスへの応用}
\subsection{リング近似との関係}
\begin{frame}
    hello
\end{frame}

\end{document}